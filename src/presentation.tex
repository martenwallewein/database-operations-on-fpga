 \documentclass{beamer}

\usetheme{MagdeburgFIN}
\usefonttheme{structurebold}
\usepackage{graphicx}
\usepackage{wrapfig,lipsum}
\usepackage{float}
\usepackage{url}
\usepackage{pdfpages}
\usepackage[ngerman]{babel}
\usepackage[utf8]{inputenc}

\title{Database Operations on FPGAs}
\subtitle{Database Operations on FPGAs}
\author{Marten Wallewein-Eising}
\date{\today}
\institute{Otto von Guericke University, Magdeburg}

% Milestone II: Concept & additional literature research

\begin{document}

\begin{frame}[plain]
 \titlepage
\end{frame}

\section[Agenda]{}
\begin{frame}
	\frametitle{Agenda}
	\tableofcontents
\end{frame}

\section{Motivation and Introduction}
\begin{frame}
	\frametitle{Motivation}
	\includegraphics[width=1.0\textwidth]{img/complex_source.png}
	%\begin{itemize}
	%	\item Screen of complex HDL code?
	%	\item Why should we consider this stuff???
	%\end{itemize}
\end{frame}

\begin{frame}
	\frametitle{Problems with Moores Law}
	"Number of transistors in integrated circuit doubles about every two years"
	\vspace*{0.5cm}
	\begin{columns}
		\begin{column}{0.3\textwidth}
			\includegraphics[width=1.0\textwidth]{img/gordon_moore.jpg}
			Gordon Moore
		\end{column}
		\begin{column}{0.66\textwidth}
				\begin{itemize}
				\item More transistors lead to more heat
				\item Not enough to increase cores and speed
				\item Memory/Power wall in von Neumann Architecture
				\item Fast growing data amounts (times 10 every 5 years)
				\item ...
			\end{itemize}
		\end{column}
	\end{columns}

\end{frame}

\begin{frame}
	\frametitle{Possible Solutions}
	Usage of different core types in CPU
	\begin{itemize}
		\item More variability, but cores must be generalized
	\end{itemize}
	\vspace*{0.5cm}%
	Usage of highly adapted Hardware (ASIC)
	\begin{itemize}
		\item High performance, but problem specific
	\end{itemize}
	\vspace*{0.5cm}%
	Usage of reconfigurable, programmable Hardware (FPGA)
	\begin{itemize} 
		\item Mix between generalization and performance
	\end{itemize}
	%\begin{itemize}
	%	\item Usage of different core types in CPU
	%	\item Usage of highly adapted Hardware (ASIC)
	%	\item Usage of reconfigurable, programmable Hardware => FPGA
	%	\end{itemize}
\end{frame}

\begin{frame}
	\frametitle{Introduction}
	\begin{center}
		\huge What is a FPGA?
	\end{center}
	
	"FPGAs consist of a plethora of uncommitted hardware resources, which can be
	programmed after manufacturing, i.e., in the field."
	%\begin{itemize}
	%	\item Field programmably gate array
	%	\item Questions?
	%\end{itemize}
	\vspace*{3cm}
	\begin{center}
		\small \emph{Data Processing on FPGAs}, Jens Teubner and Louis Woods 
	\end{center}
\end{frame}

\begin{frame}
	\frametitle{FPGA - Basics}
	\begin{itemize}
		\item Large resources of logic gates and RAM blocks to implement complex digital computations
		\item Designed to be configured after manufacturing
		\item Configuration is generally specified using a hardware description language (HDL)
		%\item Contains:
		%	\begin{itemize} 
		%		\item Array of programmable blocks
		%		\item Hierarchy of reconfigurable interconnects
		%	\end{itemize}
	\end{itemize}
\includegraphics[width=1.0\textwidth]{img/fpga.jpg}
\end{frame}

\begin{frame}
	\frametitle{FPGA - Components}
	\begin{itemize}
		\item Lookup Tables: configurable type of logic gates
		\item Registers/Flip-Flops: Small storage types on chip
		\item BRAM: On Chip RAM storage
		\item DRAM: Off Chip RAM
		\item Hard cores: implementations of often-required functionality directly in silicon.
		\item Interconnect: wiring between all available ressources
	\end{itemize}
\end{frame}

\section{Benefits and Drawbacks}
\begin{frame}
	\frametitle{Benefits of FPGAs}
	\begin{itemize}
		\item Very low power consumption
		\item High flexibility in programming hardware components
		\item Dynamic reconfiguration at runtime
		\item High degree of data and instruction parallelism
	\end{itemize}
	\vspace*{0.3cm}
	\begin{center}
		"Custom hardware allows employing the most effective form of parallelization that best suits a given task."
	\end{center}
	\vspace*{0.1cm}
	\begin{center}
		\small \emph{Data Processing on FPGAs}, Jens Teubner and Louis Woods 
	\end{center}
\end{frame}

\begin{frame}
\frametitle{Drawbacks of FPGAs}
	\begin{itemize}
		\item Small amount of memory compared to high degree of parallelism
		\item High parallelism is very difficult to achieve
		\item Difficult implementation of complex algorithms
		\item Bottleneck moves to data transfer to the FPGA 
	\end{itemize}
	\begin{center}
		"Building essentially a tailor-made piece of hardware takes the engineer beyond what he or she is used to in the software-only world."
	\end{center}
	\begin{center}
		\small \emph{FPGAs: A New Point in the Database Design Space}, Mueller et al.
	\end{center}
\end{frame}

\section{Integration in Database Systems}
\begin{frame}
	\frametitle{Integration in Database Systems}
	
	\textbf{How to integrate a FPGA into a Database System?}
	\begin{itemize}
		\item No complex operations due to small memory
		\item High IO throughput and degree of parallelism
		\item Difficult to implement and maintain various algorithms
	\end{itemize}
	\vspace*{1cm}
	\textbf{Conclusion: Basically, execute simple operations on huge data streams in parallel}
\end{frame}

\begin{frame}
\frametitle{Integration in Database Systems}
Include graphic showing multiple integrations of FPGAs
\end{frame}

\section{Database Operations}

\subsection{Sort and Join Operations}
\begin{frame}
	\frametitle{Sorting with FPGA Reconfiguration}
	\begin{itemize}
		\item Sortline sorting, fully pipelined, only in internal memory
		\item SortTree, not fully pipelined, uses external memory
		\item Read: Energy aware SQL Query
	\end{itemize}
\end{frame}

\begin{frame}
\frametitle{BRAM-BASED FIFO MERGE SORTER}
\begin{itemize}
	\item a merge sorter that can efficiently sort larger problem sizes inside the FPGA
	Koch and Torresen [2011] evaluated the FIFO merge sorter described above. Using 98\% of the available BRAM it was possible to sort 43,000 64-bit keys (344 KB) in a single iteration, i.e., by streaming the keys through the FPGA once. .e circuit could be clocked at MHz D fclk 252 resulting in a throughput of 2 GB/s.
	\item Read: Data Processing on FPGAs
\end{itemize}
\end{frame}

\begin{frame}
	\frametitle{FPGA Framework for Join Operations}
	\begin{itemize}
		\item Predicate evaluation and hash joining on FPGA as co processor
		\item equi-join queries with hash tables for efficient lookups
		\item Test with start scheme (as used in big warehouse databases)
		\item Speedup of 11.3 times running at 200Mhz compared to a 4.4Ghz multi-core system
	\end{itemize}
\end{frame}

\subsection{Query Execution}
\begin{frame}
\frametitle{Query Execution - Glacier}
\begin{itemize}
	\item Compiler that translates from SQL to VHDL
	\item Translated to relational algebra tree, than compiled into hardware curcuits
	\item draws its predictable runtime performance from statically allocating all hardware resources at circuit compilation time
\end{itemize}
\end{frame}

\begin{frame}
	\frametitle{Query Execution - FPGA Framework}
	\begin{itemize}
		\item TODO: Search in PDFs where this 
	\end{itemize}
\end{frame}

\section{Summary}
\begin{frame}
	\frametitle{Summary}
	\begin{itemize}
		\item Highly flexible use cases
		\item Reconfiguration vs Frameworks
		\item High throughput and low power consumption
		\item Very complicated to implement complex algorithms
		\item High potential in the future
	\end{itemize}
\end{frame}

\section{Sources}
\begin{frame}
	\frametitle{Sources - Literature}
\end{frame}

\begin{frame}
\frametitle{Sources - Images}
\begin{itemize}
	\item \url{https://wyncode.co/funniest-computer-programming-memes/}
	\item \url{https://upload.wikimedia.org/wikipedia/commons/thumb/6/6b/Gordon_Moore.jpg/220px-Gordon_Moore.jpg}
	\item \url{https://opsero.com/fpga-programming/}
\end{itemize}
\end{frame}
\begin{frame}
    \frametitle{Thank you for your attention!}
\end{frame}
\end{document}
